\documentclass[11pt]{article}

% ===================== PREAMBLE =====================
\usepackage[left=0.75in,right=0.75in,top=0.75in,bottom=0.75in]{geometry}
\usepackage[T1]{fontenc}
\usepackage{lmodern}
\usepackage{amsmath}
\usepackage{xcolor}
\usepackage{enumitem}
\usepackage{graphicx}
\usepackage{setspace}

\setstretch{1.15}
\setlist[itemize]{noitemsep, topsep=2pt}
\setlist[enumerate]{noitemsep, topsep=2pt}

% Darker textbook-style blue
\definecolor{examblue}{RGB}{0,70,140}

% Blue ONLY for main concept
\newcommand{\concept}[1]{\textcolor{examblue}{\textbf{#1}}}
\newcommand{\imp}[1]{\textbf{#1}}

% No paragraph indentation, but keep vertical spacing
\setlength{\parindent}{0pt}
\setlength{\parskip}{0.6em}


% ===================== DOCUMENT =====================
\begin{document}

\concept{Whole-body homeostasis} maintains stable internal conditions using a negative feedback loop.

\bigskip

\textbf{Main Control Pathway (Numbered Boxes)}

\begin{enumerate}
  \item \concept{Reference point (set point)}  
  The desired value of a regulated variable.  
  Example: body temperature $\approx 37^\circ$C.

  \item \concept{Sensor(Receptor)}  
  Detects changes in the regulated variable.  
  Example: thermoreceptors in the skin (face, ears, fingers).

  \item \concept{Afferent neurons}  
  Carry sensory information from the sensor to the brain.

  \item \concept{Error detector}  
  Compares the measured value to the reference point.
  \begin{itemize}
    \item If the values differ, an \concept{error signal} is generated
  \end{itemize}

  \item \concept{Controller}  
  Determines the appropriate response.
  \begin{itemize}
    \item For temperature regulation: thermoregulatory center in the hypothalamus
  \end{itemize}

  \item \concept{Efferent neurons}  
  Carry control signals from the brain to the effectors.

  \item \concept{Effectors}  
  Carry out the response.
  \begin{itemize}
    \item Examples: sweat glands, blood vessels
  \end{itemize}

  \item \concept{System output}  
  The response changes the regulated variable.
  \begin{itemize}
    \item Example: sweating and vasodilation lower body temperature
  \end{itemize}
\end{enumerate}

\concept{Negative feedback} occurs when the system senses an error and tries to reduce it 

\bigskip

\textbf{Key Features of Negative Feedback}

\begin{itemize}
  \item The system works to return a variable back toward its \concept{set point}
  \item The response always \textbf{opposes} the initial change (stimulus)
  \item Maintains stable internal conditions (\concept{homeostasis})
\end{itemize}

\bigskip

\textbf{Negative Feedback Loop Steps}

\begin{enumerate}
  \item A \concept{stimulus} causes a deviation from the set point
  \item A \concept{receptor} senses the change
  \item The \concept{control center} compares the value to the set point
  \item An \concept{effector} produces a response
  \item The response reduces the original error
\end{enumerate}

\bigskip

\textbf{Important Notes}

\begin{itemize}
  \item Negative feedback does \textbf{not} eliminate the stimulus, only reduces it
  \item Most physiological regulation uses negative feedback
  \item Opposite of \concept{positive feedback}
\end{itemize}

\bigskip

\textbf{One-Line Exam Definition}

Negative feedback is a control mechanism in which the response counteracts the initial change to restore the system toward its set point.


\section*{Neuron Structure and Function}

\concept{Neuron} is a specialized cell that transmits electrical signals for communication in the nervous system.
\begin{figure}[h]
  \centering
  \includegraphics[width=0.6\textwidth]{neuron.jpeg}
  \caption{Structure of a neuron}
\end{figure}

\concept{Dendrites} receive and integrate incoming signals from other neurons.

\concept{Cell body (soma)} contains the nucleus and integrates signals received from dendrites.

\concept{Axon} transmits electrical signals away from the cell body over long distances.

\concept{Axon terminals} release neurotransmitters to communicate with other neurons or target cells.


\concept{Equilibrium potential ($E_{\mathrm{ion}}$)} depends only on the ion concentration gradient across the membrane.
It does not depend on how many channels are open.

Increasing Na$^+$ leak channels:
\begin{itemize}
  \item increases how fast Na$^+$ can move across the membrane
  \item increases membrane permeability to Na$^+$
  \item does not change Na$^+$ concentration values
\end{itemize}

\concept{Mitochondria} are organelles that produce ATP.
ATP is the energy currency used by the cell.
Most ATP is produced by oxidative phosphorylation in the mitochondria.

\concept{ATPase} is an enzyme that hydrolyzes ATP to release energy that used for active transport, movement or chemical reactions.
It breaks ATP into ADP and releases energy for cellular processes.
\[
\text{ATP} + \text{H}_2\text{O}
\;\xrightarrow{\text{ATPase}}\;
\text{ADP} + \text{P}_i + \text{energy}
\]


\concept{Pump} is a protein located in the cell membrane that moves substances across the membrane.
It uses energy (usually ATP) and can move substances against their concentration gradient.

\concept{Transcription} is the process of making RNA from DNA.
\begin{enumerate}
  \item Transcription factors bind to DNA and help RNA polymerase find the correct location
  \item RNA polymerase binds to the DNA
  \item DNA opens and RNA is synthesized
\end{enumerate}

\concept{Cell membrane} provides separation and control.
It separates the inside of the cell from the outside,
controls what enters and leaves the cell,
and allows different compositions inside versus outside the cell.
It also hosts proteins involved in signal transduction from the environment.

\concept{Ribosome} is composed of ribosomal RNA (rRNA) and proteins.
It functions as the site of protein synthesis.

\concept{Protein} is made of amino acids.
Messenger RNA (mRNA) is USED(not composed) as the template to make proteins.

\concept{RNA} includes introns, mRNA, rRNA, and tRNA.
Introns are parts of RNA that are removed before a protein is made.
Proteins, enzymes, and hormones are not RNA.

\concept{Gene expression} is the process by which information in DNA is used to make a protein.
It consists of two steps.

\begin{enumerate}
  \item \concept{Transcription}  
  Transcription converts DNA into RNA:
  \[
  \text{DNA} \rightarrow \text{RNA} \quad (\text{ATCG} \rightarrow \text{AUCG})
  \]

\item \concept{Translation}  
Translation converts RNA into protein.

\begin{itemize}
  \item RNA codons pair with complementary anticodons on tRNA to add amino acids
  \item \concept{tRNA} carries specific amino acids to the ribosome during translation by matching anticodons to mRNA codons
\end{itemize}
\end{enumerate}

\medskip
\textbf{Important clarification:}

\concept{DNA replication} is \emph{not} part of gene expression.
It copies DNA using DNA as the template:
\[
\text{DNA} \rightarrow \text{DNA}
\]

\medskip
\textbf{Template summary:}

\begin{center}
\begin{tabular}{|l|l|}
\hline
\textbf{Process} & \textbf{Template} \\
\hline
DNA replication & DNA \\
Transcription & DNA \\
Translation & mRNA \\
\hline
\end{tabular}
\end{center}

\concept{Voltage-gated Na$^+$ channels} are proteins in the neuron membrane that open when the membrane voltage changes.
They allow Na$^+$ to rush into the neuron and initiate the action potential.
\begin{figure}[h]
  \centering
  \includegraphics[width=0.6\textwidth]{voltagegatedionchannels.jpeg}
  \caption{States of voltage-gated ion channels}
\end{figure}
\\Ligand-gated channels require a chemical ligand to open.
\\Leak channels are always open.

\concept{Action potential} is a brief electrical signal that travels along a neuron to transmit information.

\begin{itemize}
  \item is TRIGGERED when the membrane potential increases past a threshold value.
  \item is CAUSED by ion flow(ion moving in and out through voltage-gated channels
  \item action potentials regenerate along the membrane, so the signal does not decay with distance
  \item \imp{long-distance signal}: they involve current flow through the entire span of the plasma membrane
\end{itemize}


\imp{Basic states of an action potential:}

\begin{enumerate}
  \item \concept{Resting state} has a membrane potential of approximately $-70\,\text{mV}$,
  with the inside of the neuron negative.
  A neuron is hyperexcitable when small Na$^+$ entry causes a large depolarization.

  \item \concept{Depolarization} occurs when voltage-gated Na$^+$ channels open.
  Na$^+$ enters the neuron and the membrane potential becomes more positive.
  This is the start of the action potential.
  A neuron is unexcitable when Na$^+$ cannot enter properly and depolarization cannot occur.

  \item \concept{Repolarization} occurs when Na$^+$ channels close and voltage-gated K$^+$ channels open.
  K$^+$ leaves the neuron and the membrane potential returns toward resting values.
  In demyelinated axons, signal transmission is slower or weaker due to loss of the myelin sheath.

  \item \concept{Return to rest} occurs when the membrane potential stabilizes back at the resting level.
\end{enumerate}

\concept{Synapse} is a junction where a presynaptic neuron communicates with a postsynaptic cell.

\bigskip

\textbf{Main Components of a Synapse (Numbered Boxes)}

\begin{enumerate}
  \item \concept{Presynaptic neuron}  
  The neuron that sends the signal.  
  Neurotransmitters are released from its axon terminal.

  \item \concept{Synaptic cleft}  
  A small gap between the presynaptic and postsynaptic cells.  
  Neurotransmitters diffuse across this space.

  \item \concept{Postsynaptic cell}  
  The cell that receives the signal.  
  Usually a dendrite or cell body of another neuron.

  \item \concept{Postsynaptic receptors}  
  Proteins on the postsynaptic membrane that bind neurotransmitters.  
  Binding opens ion channels.

  \item \concept{Postsynaptic potential}  
  A change in membrane potential caused by ion movement.
  \begin{itemize}
    \item Can be excitatory (EPSP) or inhibitory (IPSP)
    \item Size depends on how many ions move
  \end{itemize}
\end{enumerate}

\bigskip

\textbf{Types of Synapses}

\begin{enumerate}
  \item \concept{Chemical synapse}  
  Uses neurotransmitters to transmit signals.
  \begin{itemize}
    \item Electrical signals are converted to chemical signals and then back to electrical signals
    \item Neurotransmitters bind to receptors and open ligand-gated ion channels
    \item Postsynaptic electrical signals are generated by movement of \concept{extracellular} ions
    \item The resulting \concept{postsynaptic potential} is graded
   \item In a chemical synapse, the electrical signal in the postsynaptic cell is newly generated.
\begin{itemize}
  \item If many ligand-gated ion channels open, more \concept{extracellular} ions move,
  producing a larger \concept{postsynaptic potential} (\concept{gain})
  \item If fewer ion channels open, fewer ions move,
  producing a smaller postsynaptic potential (\concept{attenuation})
\end{itemize}

  \end{itemize}

  \item \concept{Electrical synapse}  
  Uses gap junctions for direct ion flow between cells.
  \begin{itemize}
    \item Ions pass directly from one cell to another
    \item Signal is transmitted electrically without neurotransmitters
    \item Very fast
    \item No gain or attenuation
  \end{itemize}
\end{enumerate}



\end{document}

